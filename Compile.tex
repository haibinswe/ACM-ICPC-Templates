\documentclass[a4paper, 12pt, twoside]{article}

\setlength{\columnsep}{0.8cm}
\usepackage{xeCJK}
\setsansfont{Microsoft YaHei Mono}
\setmonofont{Microsoft YaHei Mono}
\setmainfont{Microsoft YaHei Mono}

\setCJKmainfont{Microsoft YaHei Mono}
\setCJKsansfont{Microsoft YaHei Mono}
\setCJKmonofont{Microsoft YaHei Mono}

\usepackage{listings}
\lstset{
	language=C++,
	breaklines=true,
	tabsize=4,
	basicstyle=\sf\footnotesize,
	numberstyle=\sf\footnotesize,
	commentstyle=\sf\footnotesize,
	numbers=left,
	frame=leftline,
	escapeinside=``,
	extendedchars=false,
	showstringspaces=false
}

\usepackage{geometry}
\geometry{left=2cm,right=2cm,top=1.2cm,bottom=1.8cm,headsep=0.1cm,footnotesep=0.5cm}
\usepackage{courier}

\usepackage[
	CJKbookmarks=true,
	colorlinks,
	linkcolor=black,
	anchorcolor=black,
	citecolor=black
]{hyperref}
\AtBeginDvi{\special{pdf:tounicode UTF8-UCS2}}

\usepackage{fancyhdr}

\usepackage{sectsty}
\sectionfont{\large}
\subsectionfont{\large}
\subsubsectionfont{\large}

\usepackage{amsmath}
\usepackage{booktabs}

\usepackage{indentfirst}
\setlength{\parindent}{0em}
\begin{document}
% 页面设置
\pagestyle{fancy}
\fancyfoot[C]{}
\cfoot{\thepage}
% 封面
\newgeometry{left=1cm,right=1cm,top=1.5cm,bottom=1.5cm}
\begin{titlepage}
	\pagestyle{empty}
	
  \begin{center}
		~\\[80pt]
    \fontsize{40pt}{\baselineskip}\selectfont \textsc{\textbf{ACM/ICPC Templates}}\\[32pt]
    ~\\[20pt]
    \includegraphics[scale=.7]{./Cover.jpg}
    ~\\[20pt]
    \Huge\textbf{GYSHGX868}\\[14pt]
    \Large\textbf{Last build at \today}
  \end{center}
\end{titlepage}
\restoregeometry

\setcounter{page}{1}
\pagenumbering{Roman}
\tableofcontents
\clearpage
\setcounter{page}{1}
\pagenumbering{arabic}

% --- 设置 ---
\section{设置}
\subsection{C++头文件}
\lstinputlisting{"./Settings/C++ Header.cpp"}

% --- STL ---
\section{STL}
\subsection{STL总结}
\lstinputlisting{"./STL/STL.cpp"}
\subsection{map}
\lstinputlisting{"./STL/map.cpp"}

% --- 基本算法 ---
\section{基本算法}
\subsection{二分法}
\lstinputlisting{"./Basic Algorithm/Bisection Method.cpp"}
\subsection{三分法}
\lstinputlisting{"./Basic Algorithm/Trichotomy.cpp"}

% --- 数据结构 ---
\section{数据结构}
\subsection{离散化}
\lstinputlisting{"./Data Structure/Discretization.cpp"}
\subsection{并查集}
\lstinputlisting{"./Data Structure/Union Find.cpp"}

\subsection{树状数组}
\subsubsection{一维树状数组}
\lstinputlisting{"./Data Structure/Binary Indexed Tree/BIT with Linear Array.cpp"}
\subsubsection{二维树状数组}
\lstinputlisting{"./Data Structure/Binary Indexed Tree/BIT with Dyadic Array.cpp"}
\subsubsection{三维树状数组}
\lstinputlisting{"./Data Structure/Binary Indexed Tree/BIT with 3D Array.cpp"}
\subsubsection{一维树状数组区间更新区间查询}
\lstinputlisting{"./Data Structure/Binary Indexed Tree/BIT with Linear Array Interval Update and Query.cpp"}

\subsection{线段树}
\subsubsection{单点更新}
\lstinputlisting{"./Data Structure/Segment Tree/Point Update.cpp"}
\subsubsection{区间更新}
\lstinputlisting{"./Data Structure/Segment Tree/Interval Update.cpp"}
\subsubsection{线段树 + DFS}
\lstinputlisting{"./Data Structure/Segment Tree/ST with DFS.cpp"}
\subsubsection{二维线段树区间更新单点求和}
\lstinputlisting{"./Data Structure/Segment Tree/ST with Dyadic Array Interval Update Query Point Sum.cpp"}
\subsubsection{二维线段树单点更新求最大值最小值}
\lstinputlisting{"./Data Structure/Segment Tree/ST with Dyadic Array Point Update Query Maximum and Minimum.cpp"}

\subsection{RMQ}
\subsubsection{一维RMQ}
\lstinputlisting{"./Data Structure/RMQ/RMQ with Linear Array.cpp"}
\subsubsection{二维RMQ}
\lstinputlisting{"./Data Structure/RMQ/RMQ with Dyadic Array.cpp"}

\subsection{主席树}
\subsubsection{主席树}
\lstinputlisting{"./Data Structure/Persistent Tree/Persistent Tree.cpp"}
\subsubsection{树状数组 + 主席树}
\lstinputlisting{"./Data Structure/Persistent Tree/BIT + Persistent Tree.cpp"}

\subsection{Size Balanced Tree}
\lstinputlisting{"./Data Structure/Size Balanced Tree.cpp"}

\subsection{划分树}
\lstinputlisting{"./Data Structure/Partition Tree.cpp"}

\subsection{左偏树}
\lstinputlisting{"./Data Structure/Leftist Tree.cpp"}

% --- 字符串 ---
\section{字符串}
\subsection{KMP}
\lstinputlisting{"./String/KMP.cpp"}

\subsection{扩展KMP}
\lstinputlisting{"./String/Extend KMP.cpp"}

\subsection{字典树}
\lstinputlisting{"./String/Trie Tree.cpp"}

\subsection{AC自动机}
\lstinputlisting{"./String/Aho-Corasick automaton.cpp"}

\subsection{循环同构最小表示法}
\lstinputlisting{"./String/Minimum Notation.cpp"}

\subsection{哈希}
\subsubsection{字符串哈希}
\lstinputlisting{"./String/Hash/String Hash.cpp"}
\subsubsection{字符串矩阵哈希}
\lstinputlisting{"./String/Hash/String Matrix Hash.cpp"}
\subsubsection{哈希函数}
\lstinputlisting{"./String/Hash/Hash Function.cpp"}

% --- 动态规划 ---
\section{动态规划}
\subsection{最大子段和}
\lstinputlisting{"./Dynamic Programming/Longest Subsequence Sum with Linear Array.cpp"}
\subsection{二维最大子段和}
\lstinputlisting{"./Dynamic Programming/Longest Subsequence Sub with Dyadic Array.cpp"}
\subsection{最长上升子序列 O(n\^2) + O(nlogn)}
\lstinputlisting{"./Dynamic Programming/Longest Increasing Subsequence.cpp"}
\subsection{最长公共子序列}
\lstinputlisting{"./Dynamic Programming/Longest Common Subsequence.cpp"}
\subsection{最长公共上升子序列}
\lstinputlisting{"./Dynamic Programming/LCIS.cpp"}
\subsection{数位DP}
\lstinputlisting{"./Dynamic Programming/Digital DP.cpp"}
\subsection{01背包 + 一维数组}
\lstinputlisting{"./Dynamic Programming/01 Knapsack with Linear Array.cpp"}
\subsection{01背包 + 二维数组}
\lstinputlisting{"./Dynamic Programming/01 Knapsack with Dyadic Array.cpp"}
\subsection{完全背包}
\lstinputlisting{"./Dynamic Programming/Unbounded Knapsack.cpp"}

% -- 数论 ---
\section{数论}
\subsection{质数判断}
\lstinputlisting{"./number Theory/Judge Prime.cpp"}
\subsection{Miller Rabin}
\lstinputlisting{"./Number Theory/Miller Rabin.cpp"}
\subsection{筛法求质数}
\lstinputlisting{"./Number Theory/Sieve of Eratosthenes.cpp"}
\subsection{区间筛质数}
\lstinputlisting{"./Number Theory/Interval Sieve of Eratosthenes.cpp"}
\subsection{分解质因数}
\lstinputlisting{"./Number Theory/Prime Factorizations.cpp"}
\subsection{因数个数打表}
\lstinputlisting{"./Number Theory/Factor Number Table.cpp"}
\subsection{快速乘法}
\lstinputlisting{"./Number Theory/Quick Multiplication.cpp"}
\subsection{快速幂}
\lstinputlisting{"./Number Theory/Quick Power.cpp"}
\subsection{费马小定理求逆元}
\lstinputlisting{"./Number Theory/Inverse Fermat Little Theorem.cpp"}
\subsection{扩展欧几里得算法}
\lstinputlisting{"./Number Theory/Extend GCD.cpp"}
\subsection{扩展欧几里得算法求逆元}
\lstinputlisting{"./Number Theory/Inverse Extend GCD.cpp"}
\subsection{欧拉函数}
\lstinputlisting{"./Number Theory/Euler Function.cpp"}
\subsection{欧拉函数打表}
\lstinputlisting{"./Number Theory/Euler Function Table.cpp"}
\subsection{中国剩余定理}
\lstinputlisting{"./Number Theory/Chinese Remainder Theorem.cpp"}
\subsection{组合数打表}
\lstinputlisting{"./Number Theory/Combination Table.cpp"}
\subsection{组合数在线算法}
\lstinputlisting{"./Number Theory/Combination Online.cpp"}
\subsection{Lucas定理}
\lstinputlisting{"./Number Theory/Lucas Theorem.cpp"}
\subsection{指数循环节}
\lstinputlisting{"./Number Theory/Huge Mods.cpp"}

% --- 图论 ---
\section{图论}
\subsection{邻接表}
\lstinputlisting{"./Graph Theory/Adjacency List.cpp"}
\subsection{SPFA}
\lstinputlisting{"./Graph Theory/SPFA.cpp"}
\subsection{Dijkstra}
\lstinputlisting{"./Graph Theory/Dijkstra.cpp"}
\subsection{Dijkstra + Queue}
\lstinputlisting{"./Graph Theory/Dijkstra+Queue.cpp"}
\subsection{Dijkstra + Heap}
\lstinputlisting{"./Graph Theory/Dijkstra+Heap.cpp"}
\subsection{Floyd Warshall}
\lstinputlisting{"./Graph Theory/Floyd Warshall.cpp"}
\subsection{Kruskal}
\lstinputlisting{"./Graph Theory/Kruskal.cpp"}
\subsection{Prim}
\lstinputlisting{"./Graph Theory/Prim.cpp"}
\subsection{Prim + Heap}
\lstinputlisting{"./Graph Theory/Prim+Heap.cpp"}
\subsection{最小树形图朱刘算法}
\lstinputlisting{"./Graph Theory/Chu-Liu Algorithm.cpp"}
\subsection{树的直径}
\lstinputlisting{"./Graph Theory/Tree Diameter.cpp"}
\subsection{二分图}
\lstinputlisting{"./Graph Theory/Bipartite Graph.cpp"}
\subsection{网络流Dinic}
\lstinputlisting{"./Graph Theory/Dinic.cpp"}
\subsection{LCA Tarjan}
\lstinputlisting{"./Graph Theory/LCA Tarjan.cpp"}
\subsection{2 - SAT}
\lstinputlisting{"./Graph Theory/2-SAT_usage.cpp"}
\lstinputlisting{"./Graph Theory/2-SAT.cpp"}

% --- 数学 ---
\section{数学}
\subsection{高斯消元}
\lstinputlisting{"./Mathematics/Gaussian Elimination.cpp"}
\subsection{二进制下的高斯消元}
\lstinputlisting{"./Mathematics/Gaussian Elimination with Binary.cpp"}
\subsection{BigInteger高精度整数类}
\lstinputlisting{"./Mathematics/BigInteger.cpp"}
\subsection{Fraction分数类}
\lstinputlisting{"./Mathematics/Fraction.cpp"}
\subsection{Matrix矩阵类}
\lstinputlisting{"./Mathematics/Matrix.cpp"}
\subsection{分治法等比数列求和}
\lstinputlisting{"./Mathematics/Geometric Progression Sum.cpp"}
\subsection{自适应Simpson积分法}
\lstinputlisting{"./Mathematics/Simpson Integration.cpp"}
\subsection{Romberg积分法}
\lstinputlisting{"./Mathematics/Romberg Integration.cpp"}
\subsection{De Bruijn序列}
\lstinputlisting{"./Mathematics/De Bruijn Sequence.cpp"}
\subsection{格雷码}
\lstinputlisting{"./Mathematics/Gray Code.cpp"}
\subsection{计算表达式}
\lstinputlisting{"./Mathematics/Calculate Expression.cpp"}

% --- 博弈 ---
\section{博弈}
\subsection{巴什博奕}
\lstinputlisting{"./Game/Bash Game.cpp"}
\subsection{尼姆博弈}
\lstinputlisting{"./Game/Nim Game.cpp"}
\subsection{威佐夫博奕}
\lstinputlisting{"./Game/Wythoff Game.cpp"}

% --- 几何计算 ---
\section{几何计算}
\subsection{最近点对}
\lstinputlisting{"./Geometry/Closest Points.cpp"}
\subsection{几何计算类}
\lstinputlisting{"./Geometry/Geometry.cpp"}



















\clearpage
\end{document}